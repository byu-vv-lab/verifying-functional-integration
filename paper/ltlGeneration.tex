This section focuses on the generation of the LTL properties using the CWP state diagram. We also note in this section some requirements that the CWP must satisfy in order to be ingested correctly by the automation tool. The next section discusses the generation of the Promela model using the BPMN workflow.

The LTL properties are generated using information in the CWP state diagram. The CWP is ingested into the tool as an XML file. Currently, the tool only recognizes XML elements created using the open source library \emph{MXGraph}. This library is the backend for several public tools, including the online drawing tool \emph{draw.io}.

Additionally, the CWP state diagram must have an initial state. This state must begin with the case-insensitive string "init". This is the state that the model checker is going to assume the workflow initially resides in. In order for verification to work, the initial values of the state variables must place the workflow in the initial state. This is checked before any verification step is taken in the workflow.

We explored how the CWP states are defined as boolean expressions using the edges as atomic propositions in \secref{sec:backgroundCWP}. In order to automate the process, the edges must be machine-readable boolean logic expressions. For simplicity we chose to write the edges in valid Promela syntax. Promela boolean operators follow a style popularized by C, where "<", "<=", ">", ">=", "==" and "!=" represent "less than", "less than or equal", "greater than", "greater than or equal", "equal" and "not equal" respectively. This syntax is used on the CWP provided for \facetoface.

The expression is written visually on the diagram as an edge label. This has the advantage of being completely transparent to the viewer, along with the downside that it may be unreadable to readers lacking a background in boolean logic and may get cluttered in large expressions. This can be improved in the future by using a more user-friendly expression language such as the "Friendly Enough Expression Language" (FEEL) \cite{FEELSpecification}, and by using an additional non-visual XML attribute that carries the expression separate from the edge label itself.

LTL properties are generated by iterating through the states in a CWP diagram, as shown below:
\begin{lstlisting}[style=myPython]
    def generate_LTL(self):
        self.writeGlobalProperties()
        for state in self.cwp.states:
            self.writeStateProperties(state)
            
    def writeGlobalProperties(self):
        self.writeLine("//**********GLOBAL PROPERTIES************//")
        self.writeInaStateProperty()
        self.writeFairProperty()
        
    def writeStateProperties(self, state:CWPState):
        self.writeLine("//**********{} STATE PROPERTIES************//".format(state.name))
        self.writeExistsProperty(state)
        self.writeMutexProperty(state)
        self.writeEdgesProperty(state)
\end{lstlisting}

When generating LTL, the method \emph{writeGlobalProperties()} only gets called once, whereas the method \emph{writeStateProperties()} is called once for each state in the stored CWP's list of states.