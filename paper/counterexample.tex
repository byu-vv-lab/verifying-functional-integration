Automated model verification is only useful to us as far as the generated counterexamples are comprehensible. Without comprehensible counterexamples, the negative verification results provide very little or no insight into the source of the error. For example, the verification results for a purchase example might indicate that the property "PurchasePendingEdges" had an error and therefore the results came back negative. Without any visualization, there is very little information about where or how this property was violated.

The counterexample visualizer produces a series of annotated images for both the BPMN workflow and the CWP state diagram. Each annotated image corresponds to a single step of execution during the verification process. The visualized path is the path that triggers the violation of a property. Aside from the path of execution, there are several notable annotations, particularly on the BPMN.

First, at the top of the BPMN workflow, there is a list of the state(s) in which the BPMN currently resides. This is important to pay attention to because it will show precisely when a property is violated. For example, if any "mutex" property is violated, there will eventually be a step in execution that lists multiple states at the top of the image. This moment is the moment of failure. Additionally, if the "alwaysInAState" property fails, there will eventually be a step in which no states are listed.

Next, in the top right corner of the BPMN workflow, each of the environment variables are listed, along with their current values. This information can help a user find subtle errors involving CWP edge labels, BPMN behavior models, and more. Each variable is highlighted if its value is changed after the current BPMN step is finished.

Finally, the CWP image is annotated by coloring the current state(s). If the BPMN only resides in one state, it is colored green. If the BPMN resides in multiple states simultaneously, they will be colored red. Additionally, if on the previous step, the BPMN resided in a state from which there is not a transition to the current state, the current state will be colored red.

It is important to note that the visualizer shows the BPMN first taking a step, followed by the CWP updating. Therefore, the CWP annotated image associated with step seven indicates that the corresponding BPMN resides in the colored state(s) after it has taken its seventh step. This is also consistent with the environment variables and state annotation on the resulting BPMN diagram.

The counterexample visualizer is currently unable to elaborate on existential error - that is, any error on an existential property. The verifier can indicate that an existential property was violated, but no annotated images are generated. Logically, this makes sense. If omission is the error, how it be represented visually? There is no trail to follow like there is on errors of commission. Which activity, gateway, or flow is responsible for causing the model to not achieve some state? This is a difficult question that is left for future work.
