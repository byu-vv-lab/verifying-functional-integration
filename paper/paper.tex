\documentclass[conference]{IEEEtran}
% \IEEEoverridecommandlockouts
% The preceding line is only needed to identify funding in the first footnote. If that is unneeded, please comment it out.
\usepackage{cite}
\usepackage{amsmath,amssymb,amsfonts}
\usepackage{algorithmic}
\usepackage{graphicx}
\usepackage{textcomp}

\def\BibTeX{{\rm B\kern-.05em{\sc i\kern-.025em b}\kern-.08em
    T\kern-.1667em\lower.7ex\hbox{E}\kern-.125emX}}

\usepackage{url}
\usepackage{comment}
\usepackage{listings}
\usepackage[usenames,dvipsnames,svgnames,table]{xcolor}
\graphicspath{{./figs/}} 
\usepackage{paralist}
\usepackage{svg}
\usepackage[T1]{fontenc}
\usepackage{hyperref}
\usepackage{verbatimbox}
% \usepackage{tabularx}

\definecolor{dkgreen}{rgb}{0,0.6,0}
\definecolor{mauve}{rgb}{0.58,0,0.82}
\definecolor{light-gray}{gray}{0.88}

\lstdefinestyle{myPromela}
{frame=none,
  basicstyle=\ttfamily,  
  language=Promela,
  aboveskip=1mm,
  belowskip=1mm,
  columns=flexible,
  numbers=none,  
  breaklines=true,
  numberstyle=\tiny\color{gray},
  commentstyle=\color{dkgreen},
  stringstyle=\color{mauve},
  tabsize=2,
  morekeywords={always, eventually, until, implies, ltl},
  postbreak=\mbox{\textcolor{red}{$\hookrightarrow$}\space},
  literate={\ \ }{{\ }}1,
}

\lstdefinestyle{myGrammar}
{frame=none,
  basicstyle=\itshape,
  keywordstyle=\bfseries,
  commentstyle=\dkgreen,
  xleftmargin=2em,
  literate={->}{$\rightarrow$}{2},
  morekeywords={const, NUM, enum, bool, True, False}}

\lstdefinestyle{myPython}
{frame=none,
  basicstyle=\ttfamily,
  aboveskip=1mm,
  belowskip=1mm,
  language=Python,
  columns=fullflexible,
  numbers=none,
  breaklines=true,
  showstringspaces=false,
  numberstyle=\tiny\color{gray},
  commentstyle=\color{dkgreen},
  stringstyle=\color{mauve},
  tabsize=2,
  morekeywords={always, eventually, until, implies, ltl},
  postbreak=\mbox{\textcolor{red}{$\hookrightarrow$}\space},
  literate={\ \ }{{\ }}1,
}

\lstdefinestyle{myTXT}
{frame=none,
  basicstyle=\ttfamily,
  aboveskip=1mm,
  belowskip=1mm,
  columns=fullflexible,
  breaklines=true,
  postbreak=\mbox{\textcolor{red}{$\hookrightarrow$}\space},
}

\newcommand{\figref}[1]{Fig.~\ref{#1}}
\newcommand{\secref}[1]{Sec.~\ref{#1}}
\newcommand{\tabref}[1]{Tab.~\ref{#1}}
\newcommand{\appendixref}[1]{Appendix~\ref{#1}}

\newcommand{\ie}{\textit{i.e.}}
\newcommand{\eg}{\textit{e.g.}}
\newcommand{\etal}{\textit{et al.}}
\newcommand{\etc}{\textit{etc.}}
\newcommand{\adhoc}{\textit{ad hoc}}
\newcommand{\phware}{PHware}
\newcommand{\buynsell}{BuyN'Sell}
\newcommand{\facetoface}{Face2face}
\newcommand{\scitesection}[1]{\section{\uppercase{#1}}}

\newif\ifcomments
\commentstrue
\newcommand{\egm}[1]{\ifcomments\textcolor{orange}{egm: #1}\fi}
\newcommand{\bryce}[1]{\ifcomments\textcolor{blue}{bryce: #1}\fi}

\begin{document}

\title{
  Automating Verification of BPMN Workflow Functional Correctness Using Model Checking
}

\author{
\IEEEauthorblockN{Bryce Pierson}
\IEEEauthorblockA{\textit{Brigham Young University} \\
Provo UT, USA \\
\url{bryce.mckay.pierson@gmail.com}}
\and
\IEEEauthorblockN{Eric Mercer}
\IEEEauthorblockA{\textit{Brigham Young University} \\
Provo UT, USA \\
0000-0002-2264-2958}
}

\maketitle

\begin{abstract}
    \begin{comment}
Functional integration of human cognition and machine reasoning is an industry-wide problem where failure risks health or safety.
Differences in human versus machine functioning obscure conventional integration. 
We introduce cognitive work problems (CWP) for rigorous, verifiable functional integration. 
CWP specify the cognitive problem that integrated designs must solve. 
They are technology-neutral, abstract work objects, allowing people and computing to share and transform them in coordination.
The end-to-end method is illustrated on a system that employs AI for remote patient monitoring (RPM) during COVID-19 home care. 
The CWP specified \emph{actionable risk awareness} as the medical problem RPM must solve.
Graphical modeling standards enabled user participation: CWP as finite state machines and system behavior in BPMN. 
For model checking, the CWP’s logical content was translated to linear temporal logic (LTL) and the BPMN into Promela as inputs to the SPIN model checker. 
SPIN verified the Promela implements the LTL correctly.
We conclude this CWP-derived RPM design solves the medical problem and enhances patient safety. 
The method appears general to many critical systems.

\end{comment}

Many critical systems industry-wide demand fail-safe integration of human cognition and machine reasoning.
This integration is difficult because of inherent dissimilarity between human and machine functioning.
Existing methods for validating systems with human and machine functioning require expertise in formal verification and are labor intensive.
Additionally, error output from existing methods is exclusively text-based and demands extensive time and familiarity to comprehend.
We introduce a tool for automating verification of systems with human and machine functioning.
Participation in the automation process requires very little knowledge of the underlying model checking tools.
We demonstrate the effectiveness of the tool using examples of models including subtle-but-critical errors.
We present an extensible model to demonstrate the scalability of the automated verification tool.
We also introduce an understandable visual representation of model failures to enable quick iteration on subsequent model versions.
The visual representation is critical to understanding why a model doesn't comply with given temporal properties.
We conclude that model checking is a viable method for verifying systems that require human and machine integration.

\end{abstract}

\begin{IEEEkeywords}
    Business Process Modeling, Model Checking, Model-Based Validation, SPIN, Linear Temporal Logic, Model Transformation
\end{IEEEkeywords}

\section{Introduction}
\begin{comment}

This paper introduces a new method of \emph{model based systems engineering} (MBSE) for functional integration and verification of joint human-computer teaming on cognitive tasks. 
The original contribution to integration is a specification of the abstract, cognitive work problem (CWP) the design must solve \cite{workcentered,BERRY201615,chi2010}, represented as a declarative, computation independent model \cite{Garrido}.
The technical neutrality of this cognitive model makes it a shareable object of abstract work which different types of actors can transform, despite vastly different performance properties.
Thus, a CWP provides the means to integrate the work of a distributed cognitive system that transforms a CWP from its initial state to the goal state required from the system.

The original contribution to design verification reuses the CWP as a new type of required property for model checking to prove the correctness of integrated designs.
The capability to integrate human cognition with machine reasoning and then verify design correctness can increase safety and reliability for a wide range of critical systems.

Our proof-of-technology is illustrated by a case-study of telehealth augmented by \emph{remote patient monitoring} (RPM).
Although telehealth has rapidly gained acceptance for improving accessibility \cite{Medicare} and reducing transmission rates \cite{10.1093/jamia/ocaa067} the remote and asynchronous context of telehealth also introduces risk that patients may deteriorate before physicians can be aware and intervene \cite{10.1097/ALN.0000000000003578}.
The experience of this integrated design process makes clear the difficulty of reasoning about even simple systems without a verification tool.

Bionous \phware\textsuperscript{\textregistered} is an application under development to enhance the telehealth safety of COVID-19 patients with RPM of their vital signs \cite{Phware}.
\phware\ measures and streams seven vital signs: temperature, oxygen saturation, respiration rate, pulse and its variability, and dia/systolic blood pressures.
A single \phware\ ring or finger-clip can be used at home to sense all vital signs in a 60-90 second session, and then send them to a smartphone via Bluetooth.
A smartphone app uploads the data to AI cloud services for storage and machine-learning to assign numerical values to vitals, analyze their trends, and send alerts.
A web-based dashboard displays the trends and alerts to provide clinicians with \emph{actionable risk awareness} of their outpatients.
The streaming of vital signs and alerts serve as an early warning system to maintain clinicians’ awareness and prioritize their attention, thereby enhancing home care safety.

The following section defines \emph{actionable risk awareness} as the CWP for \phware, developed with clinicians and other experts \footnote{Ann Marie Kimball, an internationally recognized MD epidemiologist in emerging infectious disease, provided medical advice}.
\secref{sec:ltl} details translation of the CWP into \emph{Linear Temporal Logic} (LTL) \cite{10.5555/975331} properties that exactly capture the meaning of the CWP and are suitable for use with a model checker.
%LTL is a first-order predicate calculus with the ability to reason about the temporal ordering of events \cite{10.5555/975331}.
%Any viable design for vital sign RPM of COVID-19 patients must verify against each LTL property created from the CWP in order to claim it takes appropriate action relative to patient risk.
\secref{sec:workflow} presents the \phware\ workflow model in the \emph{Business Process Modeling Notation} (BPMN) \cite{BPMN} that integrates the activities of clinicians, AI cloud services, and patient-caregivers.
\secref{sec:bpmn} details the translation of the workflow into an equivalent \emph{Process Meta-language} (Promela) model for the SPIN model-checker \cite{spin} with \secref{sec:env} discussing how the inputs to the workflow are defined.
The results of the SPIN verification are given in \secref{sec:results}.

The original finite state model, the BPMN model, and full Promela model, with the LTL properties for the CWP, are in a public Github repository \cite{repo}.
The \texttt{README.md} file summarizes its content and how to use SPIN to verify the model.
Figure titles in the electronic version of this manuscript link to equivalent, but larger, more legible versions in GitHub.


% The intentions of the detailed case study are to show the viability of augmenting workflows with CWP for functional integration and model checking, and to illustrate the value for automated reasoning in any design environment.
% SPIN worked effectively and efficiently in this project in terms of performance and iterative design analysis.
% The workflows and CWP revisions were easy to verify when posted, and when SPIN found violations, the accompanying counter-examples provided critical insights to the unexpected, and often overlooked, outcomes common in asynchronous interactions.
% These insights were absolutely necessary for those working on the workflows and those working on the CWP to arrive at the presented final forms.
% The experience of this integrated design process with the model checker makes clear that it is difficult to reliably reason about even simple concurrent systems without a verification tool.
\end{comment}

Low-Code/No-Code (LC/NC) development has been growing rapidly over the last decade. A 2018 survey found that 23\% of global developers used low-code platforms and another 22\% planned to implement them within a year \cite{LowCodeSurvey}. LC/NC solutions have been applied to almost every business field. In many of these fields, there is an expectation for low failure rates due to either safety concerns such as in aeronautics \cite{aeronautics1, aeronautics2}, privacy concerns such as in healthcare \cite{healthcare1, healthcare2}, financial concerns such as in banking \cite{banking1,banking2}, or even concerns about integrity such as in voting \cite{voting1, voting2}. For this reason, there is incentive for solutions to be tested for potential vulnerabilities. Existing methods for debugging and error detection in LC/NC are mostly limited to input testing, static analysis, and symbolic execution \cite{LCTesting, BPMNSimulation1} While these tools are effective in their own right, they fall short particularly in situations with concurrency.

One of the development categories in LC/NC is visual development using process modelling. Business Process Model and Notation (BPMN)\cite{BPMN} is the primary notation for modelling processes in LC/NC visual development \cite{BPMNusage}. BPMN is a notation standardized by the Object Management Group (OMG). It is a graphical notation designed to be both user-friendly and powerful. It has a large range of functional elements that enable a visual description of complex processes with asynchronous execution.

Existing methods of verifying BPMN workflows focus primarily on verifying static properties about the structure of the workflow \cite{bpmnVerification1, bpmnVerification2, ASMverification}. These methods verify properties such as absence of deadlock, fair resource usage, and absence of invalid end states. BPMN workflows are also verified against arbitrary temporal properties in \cite{bpmnVerification3}, but the method fails to verify properties about the object state. A solution in \cite{bpmnVerification4} emphasizes the importance of automating the process of model-based verification, but is limited to testing two BPMN workflows for equivalence. In summary, existing solutions do little to verify important aspects about the work produced by BPMN workflows.

The work in this paper is built upon an existing approach to verify BPMN workflows in \cite{mercer22}, which addresses the object state of the workflow directly. This solution uses model checking to verify a BPMN against an abstract definition of its intended work, provided as a Cognitive Work Problem (CWP). A CWP defines both an object state and a transition system. The work of a CWP is considered accomplished if the object state is transformed in such a way as to follow the transition system through from initial state to a goal state; If a BPMN can accomplish this, it is said to produce the work defined by the CWP. The solution in \cite{mercer22} translates BPMNs and CWPs into Process Meta Language \cite{promelaManual} (ProMeLa) and Linear Temporal Logic \cite{ltl}  (LTL) properties, respectively. The results are input into a model checker to determine whether the BPMN accomplishes the correct work. There are still some unsolved problems, however. First, the existing method is laborious and requires a high degree of expertise in model checking and Promela to use. Second, if there is some error with the workflow during verification, the resulting counterexample is difficult to navigate, even for an expert. This makes iterative model design difficult for the user. 

The work in this paper solves both problems. We introduce an algorithm, with its implementation, to compile a BPMN workflow into a Promela model and a CWP into a set of LTL properties that capture its meaning. We solve several automated reasoning problems relating to the BPMN-to-Promela translation, the CWP-to-LTL translation, and the environment generation. We also introduce a tool that provides visual annotated counterexamples for failed properties. It allows users to step forward and backward through the counterexample path, tracking how the BPMN transforms the object state along the way and observing how that is affecting the CWP state. Our solution drastically decreases the expertise and labor necessary to successfully verify a BPMN workflow and interpret the results. 

\begin{comment}
This work progresses the path toward high-integrity LC/NC solution designs by automating a process which allows BPMN models to be verified against a CWP. The CWP has two parts, an object state and a state machine; the former connects it to the BPMN workflow model. The goal of the verification process is to test whether the BPMN workflow can manipulate the object state toward any of the final states specified by the CWP state machine. Verification must also ensure that the BPMN workflow uses only allowed transitions to reach a final state.
\end{comment}

We evaluate this method as a viable way to verify BPMN functional correctness by using it on five BPMN-CWP benchmark pairs. These benchmarks provide insight into the tool's effectiveness both in model compilation and in its ability to scale reasonably according to process modeling best practices. A survey of over 800 business process models from three companies revealed that the average number of activities in a model is under ten \cite{BPMSizes}. Therefore, this tool was created to handle BPMN models that are reasonably small and would fit on a single page. In some cases, subprocesses and call activites can inject additional complexity in models. In those cases, the parent process is verified in isolation - with each subprocess and call activity mocked to behave as expected. Then, each subprocess and call activity can be verified separately, with its own corresponding CWP. This continues recursively until the entire process of hierarchical verification is complete. We also injected a subtle-but-critical error into one of the examples to both test the tool's capacity to find the issue and also to emphasize the necessity for such a tool to reason about concurrency in a way that is difficult or impossible for most humans. We also provide two extensible examples to show how the tool can scale in two ways: parallel actors and sequential decisions. We compare the results to BPMN modelling best practices provide by top Business Process Management (BPM) companies to show that the tool scales sufficiently.

Absent in our evaluations are any usability studies. This paper is not concerned with the usability of this tool in the hands of a BPMN LC/NC designer. Our evaluation shows that verification can be done, automated and scale to designs that follow general process modelling best practices. We leave the exact operation and managing guidelines, as well as usability investigations, to future work.

Significant contributions in this paper include:
\begin{enumerate}
    \item Automation of a process for verifying BPMN workflows against CWP diagrams referencing object state variables
    \item A counterexample visualization tool that provides step-by-step information about potentially problematic execution paths through the BPMN.
\end{enumerate}

\secref{sec:simpleExample} provides a simple example of a BPMN-CWP pair and shows how model checking can allow us to prove correct implementation.
\secref{sec:background} gives a background of \cite{mercer22} and how it relates to the work in this paper.
\secref{sec:automaticGeneration} looks at the problems and solutions in the task of automating the Promela model generation.
\secref{sec:toolArchitecture} gives an overview of the architecture used in the automation tool.
\secref{sec:counterexample} introduces the counterexample visualization tool, which improves significantly on the prior methods of debugging BPMN workflow errors.
\secref{sec:eval} provides evaluation metrics and the set of examples this tool is evaluated on, and explains the unique performance characteristics of each example.

\section{Simple Example}
\label{sec:simpleExample}
We use a simple example to introduce the semantics of the CWP and BPMN models. The example describes the well-known and easily-understandable process of a face-to-face private sale transaction. Private sale transactions are characterized by two parties, the buyer and the seller. In order for the sale to take place, terms and price must be agreed upon by both parties. Then, payment must be exchanged from the buyer to the seller and a product or service must be exchanged from the seller to the buyer. To further simplify the example, this CWP uses labels indicating that the product being sold is a backpack. Aside from the labels, there is nothing preventing this CWP to be used in other private sale transactions.

The CWP accomplishes two things. First, it introduces an object state composed of a set of variables. Second, it uses that object state in a transition system that defines a problem or proposed work. The variables and possible values in the object state are inferred from the set of logical conditions attached to the edges of the transition system.

This CWP's transition system, shown in \figref{fig:PurchaseCWP}, has five states, of which two are goal states. The workflow, shown in \figref{fig:face2face_error_bpmn} is responsible for transforming the object state to the goal state using only valid transitions as defined by the CWP. This example of a face-to-face purchase will be called \facetoface~throughout the paper. The first state is in the top left corner, where the purchase has been initialized. Next, there is a state near the center encompassing terms and price negotiations. There are then states for either the purchase failing near the bottom or being agreed upon in the top right. Finally, there is a state at the bottom right that indicates the ownership of the item and payment have switched parties.

\begin{figure*}[t]
  \begin{center}
    \begin{tabular}{c}
        \includegraphics[width=\textwidth]{../figs/CWP/labelledCWP.png}
    \end{tabular}
  \end{center}
\caption{CWP for a purchase with the edges labelled}
\label{fig:PurchaseCWP}
\end{figure*}

\begin{figure*}[t]
  \begin{center}
    \begin{tabular}{c}
        \includesvg[inkscapeformat=png, width=\textwidth]{../figs/BPMN/face2face_May_5_2023_error_workflow.svg}
    \end{tabular}
  \end{center}
\caption{BPMN workflow for a face-to-face purchase with negotiations.}
\label{fig:face2face_error_bpmn}
\end{figure*}

The BPMN workflow for \facetoface~ is shown in \figref{fig:face2face_error_bpmn}. To facilitate understanding of the semantics of the diagram, we employ a token, a concept used by the creators of BPMN on page 25 of the BPMN specification \cite{BPMNSpecification}. In the authors' words, a token "will traverse the Sequence Flows and pass through the elements in the Process. [It] is a theoretical concept that is used as an aid to define the behavior of a Process that is being performed. The behavior of Process elements can be defined by describing how they interact with a token as it “traverses” the structure of the Process."

The \facetoface~BPMN workflow begins on the left with the start event \emph{Start7}. The token then flows through to task one, where the buyer and seller meet. Then, the token flows to gateway \emph{XOR1}, which acts as a joining gateway for a looping structure encountered later. Then, the token flows to \emph{task 2}, where negotiations happen for both price and terms of the purchase. Next, an exclusive OR gateway branches the flow based on whether or not the negotiations were successful. If they were, and both parties agreed on price and terms, the token can continue to \emph{task 6}, where they shake hands, and then further on to \emph{both1}, a parallel gateway. Here, the token is duplicated and flows down both outgoing paths to tasks \emph{7a} and \emph{7b} where the payment and the backpack are exchanged. Another parallel gateway, \emph{end both1}, waits for both tokens to arrive before sending a token through to \emph{Purchase Completed}, a successful end event. If negotiations were unsuccessful initially, the XOR gateway sends the token back to either \emph{task 4}, \emph{task 5}, or \emph{XOR1}, depending on whether price, terms, or both were unsuccessful. Renegotiations are then repeated until the parties either agree on both terms and price, or they decide not to retry. If they decide not to retry, the token flows to the unsuccessful end event, \emph{Purchase Failed}.

\facetoface~contains our simple-but-critical injected error. This error causes the BPMN shown in \figref{fig:face2face_error_bpmn} to fail verification against the purchase CWP shown in \figref{fig:PurchaseCWP}. The error happens because the CWP assumes that the process must involve negotiations before completing. Technically, this requires there to be at least one step in the BPMN workflow where one of either \emph{terms} or \emph{price} is not set to \emph{pending}. Additionally, both must not be set to \emph{accept} or \emph{noRetry} immediately following the \emph{Init Purchase Pending} state.  In the erroneous BPMN workflow, it is possible for both the price and the terms to be successful in a single atomic step, essentially skipping the \emph{Negotiations} state. Since a transition does not exist in the CWP between \emph{Init Purchase Pending} and \emph{Purchase Agreed}, this causes an error during verification. More discussion on this error and its fix is found in \appendixref{sec:evaluationExamples}.

This error was automatically found by the tool presented in this paper. The source code for this tool can be found at \cite{sourceCode}. Code snippets used throughout the paper are in reference to \facetoface.


\section{Background}
\label{sec:background}
The work in \cite{mercer22} accomplishes a few things that are necessary background components for this work. First, it introduces a method for translating a BPMN diagram into verifiable Promela code. Second, it introduces a way of interpreting a CWP as a set of LTL properties and defines this method rigorously. Lastly, it uses a practical example in healthcare to show that the Promela and LTL can be used together to verify the integration of human and machine cognition in the original BPMN.

The method for generating Promela from a BPMN diagram is similar to a petri-net design. We imagine that the point of execution in the BPMN is a token in the petri-net. The resulting Promela model has one primary "while" loop with a set of nondeterministic options for execution. Each option corresponds to a possible movement of the token. The token is consumed in a node before being created in a following node. Then, execution returns to the primary loop. Because of the nondeterministic nature of Promela and its verifier SPIN, this process can create the comprehensive state space for the BPMN.

The CWP is interpreted as LTL properties by breaking it into two sets: global and state properties. The global properties are:
\begin{enumerate}
    \item The system is always in one of the states
    \item The system eventually reaches one of the goal states
\end{enumerate}

State properties are generated for each state in the CWP, and ensure the following:
\begin{enumerate}
    \item The state is reachable in the system
    \item When the system is in a state, it is only in that state
    \item When the system moves from the current state, it only moves to a state that has a transition from the current state
\end{enumerate}

A rigorous mathematical definition of this process can be found in **CITE APPENDIX**

These processes are applied to a medical practice centered around a device for at-home COVID-19 care. The results showed that the process verified against a provided CWP and therefore, integrated the human and machine actors correctly.

\subsection{CWP}
\label{sec:backgroundCWP}

The CWP is interpreted as LTL properties by breaking it into two sets: global and state properties. The global properties are:
\begin{enumerate}
    \item The system is always in one of the states
    \item The system eventually reaches one of the goal states
\end{enumerate}

State properties are generated for each state in the CWP, and ensure the following:
\begin{enumerate}
    \item The state is reachable in the system
    \item When the system is in a state, it is only in that state
    \item When the system moves from the current state, it only moves to a state that has a transition from the current state
\end{enumerate}

The CWP states are defined as Boolean expressions using the edges as atomic propositions. The system is said to reside in a state when the following expression is true: 

\[(I_0 \land I_1 \land \ldots \land I_n) \land \neg (O_0 \lor O_1 \lor \ldots \lor O_m)\]

where $I_0, I_1 \ldots I_n$ are incoming edges and $O_0, O_1 \ldots O_m$ are outgoing edges. The number of incoming edges is $n$ and the number of outgoing edges is $m$. 

Using the edge labels A-E from \figref{fig:PurchaseCWP}, the definition of the \lstinline[style=myPromela]{Negotiations} state looks like this:

\begin{lstlisting}[style=myPromela]
Negotiations =
    (
        (EdgeA)
        &&
        (! (EdgeC || EdgeD))
    )
\end{lstlisting}

Once each of the states is property defined, we can begin to construct the LTL properties using these states as atomic propositions. The complete mathematical definition for each property will not be given in this paper, but an example with state property number two, or mutual exclusion, will be shown for the Negotiations state in the purchase CWP.

The mutual exclusion property ensures that while the system resides in a particular CWP state, it only resides in that state, and does not reside in any other state in the CWP simultaneously. The property ensures that each state is completely distinct from each other state. This is what the LTL property looks like for Face2Face:

\begin{lstlisting}[style=myPromela]
ltl NegotiationsMutex {
    (
        always (
            Negotiations implies (
                Negotiations
                && (! Init_Purchase_Pending)
                && (! Purchase_Failed)
                && (! Ownerships_Switched)
                && (! Purchase_Agreed)
            )
        )
    )
}
\end{lstlisting}

The other properties look similar. A special property is the second global property, ensuring that a goal state is eventually reached. We call this property \emph{fair}. Satisfying the fair property is a prerequisite for state properties 1 and 3: the check for existence of a state and the constraint on allowable transitions from a state.

The fairness property is an insurance that the system eventually reaches a goal state. If this property finds an error, that error is an existential witness, indicating that there is a path leading to a goal state. This property is additionally used as a precondition for several other properties.

Fairness is used primarily to assist the model checker in avoiding infinite loops. It is also used as a precondition for other properties. We are essentially telling the model checker that if a path doesn't lead to a goal state, it shouldn't spend any resources checking whether it satisfies other LTL formulae.



\subsection{BPMN}
\label{sec:backgroundBPMN}
The method for generating Promela from a BPMN diagram follows token-based design that is referenced in the BPMN specification \cite{BPMNSpecification} discussed earlier in reference to Face2Face. To explain this design, we need to define tokens, places, actions, and triggers. Tokens are conceptual items that "traverse the Sequence Flows and pass through the elements in the Process." \cite{BPMNSpecification} Places are elements in the workflow in which a token may reside and include events, gateways, activities, and flows. Some places have associated actions. An action is composed of three steps. First, it consumes some amount of tokens. Then, it may or may not transform the object state. Last, it produces tokens. Each action has an associated trigger. A trigger is a condition about presence of tokens in specific places. For many actions the trigger is having one token present at an incoming flow. Other elements may require additional tokens, such as on an incoming message or, in the case of some join gateways, on multiple incoming flows. Once a trigger is true, the associated action is added to a set of available actions. At each step in execution, a single action is chosen nondeterministically.

To simulate this behavior in Promela, we use a while loop. The following code shows the template used: 

\begin{lstlisting}[style=myPromela]
active proctype name() {
  /* Add tokens to initial elements */
  do
  :: /* element_0 trigger */ ->
    /* element_0 actions */
  :: /* element_1 trigger */ ->
    /* element_1 actions */
  /* ... */  
  :: /* element_n trigger */ ->
    /* element_n actions */
  od
}
\end{lstlisting}

Each process in the Promela model has one primary "while" loop with a set of nondeterministic execution options. Each option corresponds to a possible action. At each step, the verifier chooses one action to perform. Then, execution returns to the primary loop. Because of the nondeterministic nature of Promela and its verifier SPIN, this process can create the comprehensive state space for the BPMN. A state in the BPMN is characterized both by the location of tokens and the value of object state variables. 

\section{Automated Generation of the Promela Model}
\label{sec:automatedGeneration}
\begin{comment}
Automatic translation from CWP and BPMN to LTL properties and Promela code, respectively, is not trivial. First, the edges of the CWP must follow a consistent expression language in order to be interpreted automatically. Next, the object state must be clearly defined. Also, several implied BPMN semantics lead to ambiguous workflow structures, requiring additional automated reasoning. Next, the complete conditions of decision gateways in the workflow must be expressed somewhere. Finally, the environment for running the workflow must be defined, including variables and behavior models for BPMN activities.
\end{comment}

\figref{fig:NewSolutionRoadmap} shows a diagram of our improved solution for verifying BPMNs against CWPs. It is an improvement upon the existing solution described in \secref{sec:background}. Our additions to the solution are colored teal and dashed. One of the primary advantages of the new solution is that almost the entire translation and verification processes are automated.

One noteworthy goal of our automated generation was to create idiomatic code that is relatively understandable by someone familiar with the Promela language. We made this a goal because some errors can be diagnosed and solved most quickly by direct inspection of the Promela model. Therefore, there is overhead, some of which is found in code snippets throughout the paper, to make the generated Promela more readable. For example, we use parts of task labels in their unique identifiers (as opposed to standard BPMN encoding, which assigns random alphanumeric strings.) Additionally, because we don't have access to a code formatter for Promela, we keep track of indentation and whitespace throughout the code generation process. Even though these factors are not critically important to our solution's technical success, we maintain that it is a significant goal. One important piece of future work is a user study evaluating the readability of the code generated as part of this project.

The first addition is an input file that provides a list of variables, their initial values, and their domains. This helps define the object state, which is a unique state of execution for the BPMN. The next addition is another input file that provides details on the manner in which the workflow modifies state variables at each of its activities. Details on each of the additional inputs are given in \secref{sec:objectStateDefinition} and \secref{sec:behaviorModels}

The next two additions are the LTL generation and the Promela generation. Previously, these two steps were performed manually. In this tool, both are automated. These automation processes are discussed in both \secref{sec:ltlGeneration} and \secref{sec:promelaGeneration}. More details on the architecture of the code generation tools are given later in \secref{sec:toolArchitecture}.


\subsection{State Variable Declaration}
\label{sec:objectStateDefinition}
Defining an object state for the BPMN and mapping its variables to those referenced in the CWP is a necessary prerequisite for generating verifiable Promela models. The object state for a BPMN is defined as both the set of mutable variables with their current values, as well as the locations of tokens present in the model.

The set of variables, both mutable and constant, is provided through an additional input TXT file. The following is a grammar for a line in the file:
\begin{comment}

%
{\small
\begin{lstlisting}[style=myTXT]
const byte paymentAmount paymentAmount 253
const byte belowPaymentAmount belowPaymentAmount 252
const byte noRetryPayment noRetryPayment 254
const enum INIT INIT
const enum other other
const enum buyerName buyerName
const enum sellerName sellerName
const enum agreed agreed
const enum failed failed
const byte pendingPayment pendingPayment 255
const enum pending pending
const enum noRetry noRetry

enum terms terms pending [agreed, failed, pending, noRetry]
enum backpackOwner backpackOwner sellerName [buyerName, sellerName]
enum paymentOwner paymentOwner buyerName [sellerName, buyerName]
byte paymentOffered paymentOffered pendingPayment [252-255]
\end{lstlisting}
}
%

\end{comment}

\vspace{1em}
\begin{tabular}{p{5em} p{1em} p{16em}}
        Line   & $\rightarrow$ & ConstEntry | VarEntry \\
        ConstEntry & $\rightarrow$ & \textbf{const} Numtype \textcolor{blue}{\textit{ID ID NUM}} \newline
                    | \textbf{const} \textbf{enum} \textcolor{blue}{\textit{ID ID}} \newline
                    | \textbf{const} \textbf{bool} \textcolor{blue}{\textit{ID ID}} \textbf{True} \newline
                    | \textbf{const} \textbf{bool} \textcolor{blue}{\textit{ID ID}} \textbf{False} \\
        VarEntry  & $\rightarrow$ & Numtype \textcolor{blue}{\textit{ID ID INIT Domain}} \newline
                    | \textbf{enum} \textcolor{blue}{\textit{ID ID INIT Domain}} \newline
                    | \textbf{bool} \textcolor{blue}{\textit{ID ID INIT}} \\
        Numtype   & $\rightarrow$ & \textbf{byte} | \textbf{short} | \textbf{int} \\
\end{tabular}
\vspace{1em}

Bolded words are literal terminals. Words colored blue and italicized are terminals that require a bit more explanation.

Lines beginning with the keyword "const" define constants. Each other line defines an object state variable. 

Next, the type is listed. Possible data types are "enum", "byte", "short", "int", and "bool". When parsing the conditions in the BPMN workflow and the CWP state diagram, a run-time error will occur if there are typing mismatches or if a non-numerical type is used in an inequality. 

The typing is followed by both the CWP and BPMN reference names. For each constant and variable, both a CWP representation and a BPMN representation are given as an ID. An ID is any alphanumeric string beginning with a letter. The string may also contain some special characters. This allows the author to map variables between two diagrams that may not have been developed synchronously. In the case of \facetoface, it made sense to make the variable names match precisely between the two diagrams. But we acknowledge that this may not always be possible or desirable. 

After the CWP and BPMN reference names, non-enum constants are given a value. This value can either be a boolean or a whole integer, denoted by \emph{NUM} in the grammar.

If the line defines an object state variable, it provides an initial value, shown in the grammar as \emph{INIT}. The initial value must either be a type-matching constant or a literal value. Every state variable must be assigned an initial value. Additionally, it is vital that these initial values result in the object state residing in the \emph{init} CWP state. Otherwise, verification will fail immediately before any BPMN transitions are taken.

Finally, for state variables, a \emph{Domain} is provided. This is a set of possible values are given in square brackets. This set is presented as a comma-separated list of values or ranges. This information has two purposes. First, it provides a loose domain for each variable when the model checker performs an exhaustive state space search. In most situations the domain must be restricted further. Details on this will be given in the next section, \secref{sec:behaviorModels}. The second purpose is error detection. During verification, if an object state variable ever has a value outside of its range of possible values, an error will immediately be reported, even if the error does not cause an issue with any particular property. Domains are not provided for boolean type object state variables.

Location of tokens is represented in Promela as a bit for each possible token location. If a token is present, the bit is set to "True". If there is no token, it is "False". Locations of tokens effectively capture the point of execution the BPMN workflow is currently in. Where tokens are located is the second major component of the object state definition.

\begin{comment}
We acknowledge that restricting each location to a single token is a fairly tight restriction on the generic petri-net structure, but it is sufficient for most BPMN workflows. BPMN is designed to be easily readable and understandable at a glance, while a complex and unintuitive workflow structure is usually required to result in multiple tokens at the same location.
\end{comment}

With a concrete object state defined, the next step toward automation is expressing how and where these state variables are transformed in the BPMN workflow. This step is discussed in the next section.

\subsection{Behavior Models}
\label{sec:behaviorModels}
The next step to generating a state space for the BPMN diagram is defining how and where the state should evolve. We call this the behavior models. Each task in the BPMN has a behavior model defining how it can transform the state. The following is an example of the behavior model for task 2 in Face2Face: 

\begin{lstlisting}[style=myPromela]
inline T2_BehaviorModel(){
    if
        :: true -> terms = agreed
        :: true -> terms = failed
    fi
    if
        :: true -> paymentOffered = paymentAmount
        :: true -> paymentOffered = belowPaymentAmount
    fi
    updateState()
}
\end{lstlisting}

There are a few interesting things to note here. First, the semantics of an "if" statement in Promela are different than in a traditional programming language. In Promela, an "if" statement is followed by a series of options, represented by the double colon followed by a "guard" statement. In this case, ":: true" for each option. Only one option is ever executed, but the choice is nondeterministic. During verification, SPIN will explore every possible path, so when it reaches the first "if" statement in this behavior model, a new group of state spaces will be generated where "terms" is set to "agreed" and another group where "terms" is set to "failed". Each of these groups then splits again based on the result of the second "if" statement.

Therefore, behavior models are not meant to precisely mirror the underlying business logic of a process. Instead, the behavior models simulate all possible outcomes of the business logic without the need for representing every input factor. For example, there are a lot of factors that may determine whether negotiations end in agreed terms or failed terms. Factors such as price flexibility of the seller, urgency of the purchase for the buyer, inventory levels of the product, and more can affect the outcome of the negotiations. During model checking, the behavior models simulate all possible outcomes of a task while masking all the input factors external to the BPMN model. This does not mean that every outcome is always possible. Options in behavior models can be conditional on other state variables. For example, it may be reasonable to add in a third option in Task 2's behavior model as follows:


\begin{lstlisting}[style=myPromela]
inline T2_BehaviorModel(){
    if
        :: true -> terms = agreed
        :: true -> terms = failed
        :: terms == failed -> terms = noRetry
    fi
    if
        :: true -> paymentOffered = paymentAmount
        :: true -> paymentOffered = belowPaymentAmount
        :: paymentOffered == belowPaymentAmount -> paymentOffered = noRetryPayment
    fi
    updateState()
}
\end{lstlisting}

In this version, "terms" and "paymentOffered" can be set to "noRetry" and "noRetryPayment" respectively, but only on the condition that negotiations had failed already previously.

The final line of every behavior model is updating the state. We provide sections of this macro here to illustrate its effect:

\begin{lstlisting}[style=myPromela]

/**********UPDATE STATE INLINE************/
inline updateState() {
    d_step {
        if
        :: (terms != pending || paymentOffered != pendingPayment) -> 
            EdgeA = 1
        :: else -> 
            EdgeA = 0
        fi
        /****MORE EDGE ASSIGNMENTS***
        .
        .
        */
        if
        :: (paymentOwner == buyerName &&backpackOwner == sellerName ) -> 
            EdgeE = 1
        :: else -> 
            EdgeE = 0
        fi
        Init_Purchase_Pending = 
            (
                (EdgeE)
                &&
                (! (EdgeA))
            )
        /****MORE STATE ASSIGNMENTS***
        .
        .
        */
        Negotiations = 
            (
                (EdgeA)
                &&
                (! (EdgeC || EdgeD))
            )
        assert(termsInvariant)
        assert(backpackOwnerInvariant)
        assert(paymentOwnerInvariant)
        assert(paymentOfferedInvariant)
    }
}
\end{lstlisting}


This macro uses the new state variable values to assign a bit value for each edge in the CWP. Then, each edge value is used to assign a bit value for each CWP state. Finally, we assert that each state variable is assigned a value within its given domain. Additionally, one of the more significant improvements on the original work in \cite{mercer22} is a shift from "atomic" blocks to "d\_step" blocks. This change informs the SPIN model checker of blocks of code that will execute deterministically, and drastically reduces the size of the state space. This change will be discussed in more detail during evaluation.

A set of behavior models should be provided to the automation tool as a Promela file. This file must contain a behavior model inline for each non-flow element in the BPMN. If this file doesn't exist when beginning verification, the tool will halt and write a skeleton file for you. This skeleton file is helpful as it automatically generates each inline definition. It also populates each inline with an extremely loose manipulation of the state variables. That is, it sets each variable to each possible value in every single task. Most likely, this is not the desired behavior, so the author is encouraged to modify the contents of each behavior model to better represent the business logic.

Now that we understand how the object state is defined and where and how it evolves, we take a look at the automated generation of the LTL properties from the CWP diagram and the Promela model from the BPMN workflow. These translations will be discussed in the following two sections.



\subsection{LTL Generation}
\label{sec:ltlGeneration}
This section focuses on the generation of the LTL properties using the CWP state diagram. We also note in this section some requirements that the CWP must satisfy in order to be ingested correctly by the automation tool. The next section discusses the generation of the Promela model using the BPMN workflow.

The LTL properties are generated using information in the CWP state diagram. The CWP is ingested into the tool as an XML file. Currently, the tool only recognizes XML elements created using the open source library \emph{MXGraph}. This library is the backend for several public tools, including the online drawing tool \emph{draw.io}.

Additionally, the CWP state diagram must have an initial state. This state must begin with the case-insensitive string "init". This is the state that the model checker is going to assume the workflow initially resides in. In order for verification to work, the initial values of the state variables must place the workflow in the initial state. This is checked before any verification step is taken in the workflow.

We explored how the CWP states are defined as boolean expressions using the edges as atomic propositions in \secref{sec:backgroundCWP}. In order to automate the process, the edges must be machine-readable boolean logic expressions. For simplicity we chose to write the edges in valid Promela syntax. Promela boolean operators follow a style popularized by C, where "<", "<=", ">", ">=", "==" and "!=" represent "less than", "less than or equal", "greater than", "greater than or equal", "equal" and "not equal" respectively. This syntax is used on the CWP provided for \facetoface.

The expression is written visually on the diagram as an edge label. This has the advantage of being completely transparent to the viewer, along with the downside that it may be unreadable to readers lacking a background in boolean logic and may get cluttered in large expressions. This can be improved in the future by using a more user-friendly expression language such as the "Friendly Enough Expression Language" (FEEL) \cite{FEELSpecification}, and by using an additional non-visual XML attribute that carries the expression separate from the edge label itself.

LTL properties are generated by iterating through the states in a CWP diagram, as shown below:
\begin{lstlisting}[style=myPython]
    def generate_LTL(self):
        self.writeGlobalProperties()
        for state in self.cwp.states:
            self.writeStateProperties(state)
            
    def writeGlobalProperties(self):
        self.writeLine("//**********GLOBAL PROPERTIES************//")
        self.writeInaStateProperty()
        self.writeFairProperty()
        
    def writeStateProperties(self, state:CWPState):
        self.writeLine("//**********{} STATE PROPERTIES************//".format(state.name))
        self.writeExistsProperty(state)
        self.writeMutexProperty(state)
        self.writeEdgesProperty(state)
\end{lstlisting}

When generating LTL, the method \emph{writeGlobalProperties()} only gets called once, whereas the method \emph{writeStateProperties()} is called once for each state in the stored CWP's list of states.

\subsection{Promela Generation}
\label{sec:promelaGeneration}
This section examines the mechanism for automated translation from BPMN workflows into verifiable Promela models. We look at requirements for the workflow to be correctly ingested as well as the translation implementation itself.

The BPMN diagrams are ingested into the translation tool as .bpmn files, which is a file format that contains XML data. The .bpmn format was standardized by Camunda, a Business Process Management company and investor in the BPMN standard. Diagrams can be exported into .bpmn using either the open-source online editor at bpmn.io, or Camunda's free desktop modeller. Both of these tools use the open-source project bpmn-js as a backend.

In line with the desire to create code that can easily be traced back to the source diagram, the automation tool requires that each activity in the workflow be prepended with a unique alphanumeric string followed by a hyphen. This allows the code to reference that activity in a concise, understandable way - "T$N$", where $N$ is the alphanumeric string. Verification will fail to execute if $N$ is not a unique value for each task. Full labels are used as unique identifiers for all other elements because gateways and events tend to have shorter labels. In a similar fashion, if two gateways or events have identical labels, verification will fail to execute. If an element has no label, a random unique identifier is assigned to it.

In order to facilitate easy addition of attributes and traversal of the BPMN, we are using an intermediate Python representation before generating any Promela code. This Python representation is a graph structure, where flows are edges of the graph and elements are nodes.

We note that there is no syntactic check on the workflow's structure. It is the author's responsibility to ensure valid BPMN is provided. Several free BPMN editors assist authors in creating a valid structure. 

We use a visitor pattern to distinguish the BPMN structure from the code generation algorithm. The code generator visitor traverses the structure, visiting each element. An element could be any BPMN element, such as an event, task or flow. When the code generator visits an element, various code snippets are generated based on the type of element. We start at the top, looking at the visitation of the model:
\begin{lstlisting}[style=myPython]
    def visit_model(self, element:Model) -> None:
        initLines = "init {\n"
        initLines += "\tatomic{\n"
        initLines += "\t\tupdateState()\n"
        for process in element.processList:
            initLines += "\t\trun {}()\n".format(process.name)
        initLines += "\t}\n"
        initLines += "}\n\n"
        self.writeInitLines(initLines)
        for process in element.processList:
            process.accept(self)
        for place in self.flowPlaces:
            self.writePlacesLines("bit {x} = 0".format(x = str(place)))
\end{lstlisting}

First it writes the \emph{init} process, which is the process used in Promela to initialize other processes. The \emph{init} process first updates the state to ensure we are in a valid CWP state to begin. Then, it runs a separate process for each process in the BPMN workflow. After the \emph{init} process is written, the visitor will visit each of the processes in the model. Here we look at the behavior of a process when it is visited:
\begin{lstlisting}[style=myPython]
    def visit_process(self, element:Process) -> None:
        self.writeWorkflowLines("proctype {x}() {{".format(x = element.name))
        self.workflowIndent += 1
        self.writeWorkflowLines("pid me = _pid")
        for startEvent in element.startStateList:
            if not startEvent.inMsgs:
                self.writeWorkflowLines("putToken({x})".format(x = startElement.label))
        self.writeWorkflowLines("do")
        for startEvent in element.startStateList:
            startEvent.accept(self)
        self.writeWorkflowLines("od")
        self.workflowIndent -= 1
        self.writeWorkflowLines("}")
\end{lstlisting}

The first thing it does is identify its process identification number (pid). Then, it puts tokens in each of its start elements. Finally, the visitor visits each start event.

Next we'll show the default behavior for the visitor at an element:

\begin{lstlisting}[style=myPython]
    def visit_element(self, element:Element) -> None:
        if element.seen:
            return
        element.seen = True
        self.genPlaces(element)
        self.genActivationOption(element)
        for flow in element.outFlows:
            flow.accept(self)
\end{lstlisting}

A few interesting things happen. First, a place is generated for a token to reside in eventually. Also, an activation option is generated, which adds an entry into the large "do" loop mentioned in \secref{sec:backgroundBPMN}. After this, each flow originating at this element is visited. This default behavior is most notably used by non-start and non-end events. Other element types behave similar to the default, with slight variations.

The following is the code for visiting a flow:

\begin{lstlisting}[style=myPython]
    def visit_flow(self, element:Flow) -> None:
        if element.seen:
            return
        element.seen = True
        target = element.toNode
        target.accept(self))
\end{lstlisting}

The flow is marked as seen and then the target of the flow is visited. 

\begin{comment}
Because BPMN does not have rigorous execution semantics defined, there are some situations in BPMN where the function of an element is implied by the structure in which it is used. For example, gateways can either be forking gateways (with multiple outgoing flows) or joining gateways (with multiple incoming flows). This distinction is not addressed directly in the BPMN syntax. We handle this by splitting some traditional BPMN element classes into multiple element classes in the python intermediate representation. There is in the python representation, for example, both a "parallelGatewayJoinNode" class and a "parallelGatewayForkNode" class. This distinction is important because 
\end{comment}

In our use of BPMN workflows, some of the flow labels carry significant information necessary to the control flow. These labels are on flows emanating from branching XOR gateways. We also have requirements for the content of these flows: the automation tool enforces syntactically correct Promela boolean logic. If the automation tool find a flow emanating from a branching XOR gateway is not valid boolean logic, it will throw an exception.

Therefore, we parse and perform both a syntactic and type check on the labels. The expected syntax is identical to the syntax mentioned at the beginning of \secref{sec:objectStateDefinition}. To summarize, it is basically the set of boolean operators found in C-type programming languages. Each edge can contain exactly one expression. The type check verifies that all the variable typings are valid in an expression. For example, non-numeric types should not be compared with numeric types and non-numeric types should never be involved in inequalities.


\section{Error Visualization}
\label{sec:counterexample}
Automated model verification is only useful to us as far as the generated counterexamples are comprehensible. Without comprehensible counterexamples, the negative verification results provide very little or no insight into the source of the error. For example, the verification results for a purchase example might indicate that the property "PurchasePendingEdges" had an error and therefore the results came back negative. Without any visualization, there is very little information about where or how this property was violated.

The counterexample visualizer produces a series of annotated images for both the BPMN workflow and the CWP state diagram. Each annotated image corresponds to a single step of execution during the verification process. The visualized path is the path that triggers the violation of a property. Aside from the path of execution, there are several notable annotations, particularly on the BPMN.

First, at the top of the BPMN workflow, there is a list of the state(s) in which the BPMN currently resides. This is important to pay attention to because it will show precisely when a property is violated. For example, if any "mutex" property is violated, there will eventually be a step in execution that lists multiple states at the top of the image. This moment is the moment of failure. Additionally, if the "alwaysInAState" property fails, there will eventually be a step in which no states are listed.

Next, in the top right corner of the BPMN workflow, each of the environment variables are listed, along with their current values. This information can help a user find subtle errors involving CWP edge labels, BPMN behavior models, and more. Each variable is highlighted if its value is changed after the current BPMN step is finished.

Finally, the CWP image is annotated by coloring the current state(s). If the BPMN only resides in one state, it is colored green. If the BPMN resides in multiple states simultaneously, they will be colored red. Additionally, if on the previous step, the BPMN resided in a state from which there is not a transition to the current state, the current state will be colored red.

It is important to note that the visualizer shows the BPMN first taking a step, followed by the CWP updating. Therefore, the CWP annotated image associated with step seven indicates that the corresponding BPMN resides in the colored state(s) after it has taken its seventh step. This is also consistent with the environment variables and state annotation on the resulting BPMN diagram.

The counterexample visualizer is currently unable to elaborate on existential error - that is, any error on an existential property. The verifier can indicate that an existential property was violated, but no annotated images are generated. Logically, this makes sense. If omission is the error, how it be represented visually? There is no trail to follow like there is on errors of commission. Which activity, gateway, or flow is responsible for causing the model to not achieve some state? This is a difficult question that is left for future work.


\section{Tool Architecture}
\label{sec:toolArchitecture}
The source code for the automation tool can be found \href{https://bitbucket.org/byu-vv/bpmn-cwp-verification/src/main/code/}{here}. The primary language is Python 3, with a small amount of shell scripting as well. 

\section{Evaluation}
\label{sec:eval}
This tool has been used to verify five BPMN-CWP pairs. The effectiveness of the tool will be determined by a number of factors, including model generation time, total verification time, and visual counterexample generation time. Two controllable, scalable working examples will be used to measure both horizontal (many decisions) and vertical (many synchronous actors) scalability of this model checking approach.

\egm{I would take the most simple buy-n-sell example and move it to a section after the introduction titled: Simple Example. That section should walk through the CWP, the BPMN, any other additional information needed for the translation, a verification run with a property that fails, the counter-example, the change, and then the passing verification run. Nothing of the implementation details is discussed. The interface is being shown along with exactly what the user provides and how the CWP and BPMN are connected together.}

Figure 1 shows the BPMN of the first example, a face-to-face purchase. The Buyer and seller are condensed into a single actor for simplicity in this model. The purchase process involves negotiation periods for both the price and the terms. Once the price and terms are agreed upon, the model can proceed to the exchange and then exit. Figure 2 shows the CWP for this purchase. The CWP has 5 states, of which 2 are goal states. It is the workflow's responsibility to either send the state of the CWP to "Purchase Failed" or to "Ownerships Switched". Note that even though "Purchase Failed" might not sound like a goal state, it is for purposes of verifying eventual termination.


\egm{There needs to be discussion, at the beginning, that gives an overview of the evaluation in terms of what it is trying to show and how it goes about showing that. The reader then understands why each of the example is included and discussed. What measures are we going the show with each example? Verification time? Model development time? Model size? I'm not sure what is of value.}

Figure 3 shows the BPMN of the second example, an asynchronous remote purchase system named "BuyN'Sell". This BPMN splits the buyer and seller into different parties, as well as introducing some reasoning about fulfillment of the order. BuyN'Sell shares a CWP with the first example. This was done on purpose to show the power of CWP. Because it shouldn't reason at all about the context or the actors in the work, it can be used to verify multiple "purchase" BPMNs with drastically different performance characteristics.

Figure 4 shows the BPMN of the third example, the original example from \cite{mercer22}. In this example, a patient diagnosed with COVID-19 is being monitored remotely. Figure 5 shows the corresponding CWP. When we verify the BPMN against this CWP, we are ensuring that the patient is never incidentally mistreated in the BPMN according to the readings from the remote monitoring device.

Figure 6 shows the BPMN of the fourth example. In this example, we show the scalability of the automation and verification process. This example contains a single actor with multiple decisions to make sequentially. We call this horizontal scaling. Figure 7 shows the corresponding CWP. The CWP here has just two states, only requiring that the BPMN execute to completion.

Figure 8 shows the BPMN of the fifth example. This is also an example to show scalability. In this example, there are multiple actors, all with a single decision to make. It is important to note that all of the actors interact with a single global state variable here. This type of scaling we will call vertical scaling. This example shares a CWP with the previous one.

TODO: For each example:
\begin{itemize}
    \item Insert Images of CWP/BPMN
    \item Describe the basic structure
    \item Run through an example execution path (use annotated images from counterexample visualizer)
    \item Explain why it is being used
\end{itemize}

TODO: Give computer specs and timing for each of the given examples

TODO: Detail a few of the subtle errors found by the model checking



\section{Related Work}
\begin{comment}
It is possible to translate live sequence charts to LTL \cite{KUMAR, KUMAR2009137}. These translations result in large intractable formulas whereas the work here creates several connected small formulas that are easy for SPIN. UML modeling uses \emph{synchronous observer automata} to encode and verify safety properties \cite{8906967}. The CWP can be thought of as a synchronous observer, and it is possible to express it in SPIN for safety verification, but that complicates existential properties and may result in a larger state space. Other work verifies that UML state diagrams implement their associated activity diagrams (workflows) with the NuSMV model checker \cite{7436156}. The workflows are turned into LTL, just opposite of the work here. That said, it is possible to extract a CWP from workflow models, in which case the intent is to discover what the system will accomplish.

There is some work related in translating models in the \emph{Business Process Execution Language} (BPEL) to Promela \cite{bpelToPromela}. The semantics are different and is limited in scope to web-services. BPMN choreographies have been modeled in Promela and verified with SPIN for deadlock, but choreographies ignore workflows and only model message sequencing \cite{choreography}. The translation of BPMN to Promela is inspired by existing methods for turning Petri Nets into equivalent Promela models \cite{petrinetToPromela, petrinetInspiration}. These however do not include data. The translation in this paper is also based off of early prototype translations of BPMN to Promela using message channels for synchronization \cite{bpmn2promela} whereas the translation here uses global variables to mitigate state explosion.
\end{comment}

The problem of verifying various properties of visual workflows has been addressed to various extents. In \cite{workflowVerification1}, they use Hoare logic to formally introduce and verify workflow semantics. In \cite{workflowVerification2}, they present the theoretical framework for applying propositional logic to workflow verification. In \cite{workflowVerification3}, they present a taxonomy on the state of grid workflow verification and validation. In \cite{workflowVerification4}, they present an automated tool for translation from YAWL into DVE in order to perform property verification using DiVinE, a model checker. In \cite{workflowVerification5}, they use a petri-net approach to analyze workflows for compositional and control-flow errors in workflows. In \cite{workflowVerification6}, they explore the viability of model checking in a single workflow domain, and show that it can be effective. In \cite{workflowVerification7}, they use ontologies to verify the replicability of scientific workflows.

There are several examples of efforts to apply concrete execution semantics to BPMN workflows. In \cite{BPMNExecutionSemantics1}, they present the symbolic encodings of the execution semantics of BPMN in the form of LTL. In \cite{BPMNExecutionSemantics2}, they use graph rewrite rules to formalize BPMN execution semantics and further use these rules to verify process engines. In \cite{BPMNExecutionSemantics3}, they define an abstract model for the dynamic semantics of BPMN 2.0 and use that model to implement a native BPMN 2.0 process engine. In \cite{BPMNExecutionSemantics4}, they define an ASM designed to rigorously specify the semantics of BPMN process diagrams. They also identify inconsistencies and ambiguities in the BPMN specification and attempt to fix them. In \cite{BPMNExecutionSemantics5}, they show how a subset of the BPMN specification can be given a process semantics in Communicating Sequential Processes. In \cite{BPMNExecutionSemantics6}, they provide a direct formalization of the operational semantics of BPMN in terms of Labelled Transition Systems. In \cite{BPMNExecutionSemantics7}, they propose a formal semantics in terms of a mapping to Petri Nets and implement said mapping. Each of these efforts inspired us in our own BPMN semantic assumptions.

A core component of our paper is automated Promela code generation. Similar techniques for code generation have been used in several papers. In \cite{promelaGeneration1}, \cite{promelaGeneration2} and \cite{promelaGeneration3}, they generate Promela code from Specification and Description Language (SDL) system specifications. In \cite{promelaGeneration4}, they translate UML sequence diagrams into Promela for the purpose of verifying LTL properties. In \cite{promelaGeneration5}, they perform a program-to-model transformation on C# code using the Microsoft Roslyn technique. In \cite{promelaGeneration6}, they present a case study using CINCO, a code generation framework, to rapidly construct custom graphical interfaces for multi-faceted, concurrent systems. In \cite{promelaGeneration7}, they add Promela as a supported output language to the Reo compiler.

Several papers have represented workflows as code in some way.  Some translate the workflow into a language that is used specifically for verification \cite{codeBasedModel1} \cite{codeBasedModel2}, while others use the non-visual executable counterpart to BPMN, Business Process Execution Language (BPEL). \cite{BPEL} \cite{codeBasedModel3} \cite{codeBasedModel4}

Several papers have used a petri net structure to represent workflow models. In \cite{petriNet1}, they introduce workflow management as an application domain for Petri nets. In \cite{petriNet2}, they review and classify available papers on Petri net-based modeling of workflows. In \cite{petriNet3}, timed Petri nets are used to show that certain behavioral properties of workflow processes can be verified. In \cite{petriNet4}, they verify process control specifications using state-of-the-art Petri net based analysis techniques. In \cite{petriNet5}, they use colored petri nets to assist in modeling families of workflow processes with similar process routes and logic rules.
\COMMENT{petriNet6 is an older paper on petri net application to workflow management}

The construct we use to represent the work needing to be done, the CWP, is used in several papers. This presentation uses CWPs to define the work of a patient-centered case management system \cite{CWPRW1}. In \cite{CWPRW2} and \cite{CWPRW3}, CWPs are used in conjunction with model checking to verify properties of health systems. In \cite{CWPRW4} and \cite{CWPRW5}, CWPs are used in the design process of health IT systems.

We derive a set of LTL properties using a visual state diagram. Representing LTL visually can increase its effective use in business settings dramatically. Others have also implemented methods of visually representing LTL properties. In \cite{LTLVisualization1}, LTL is extracted from UML diagrams. In \cite{LTLVisualization2}, the authors propose two alternative visual notations for specifying temporal properties. In \cite{LTLVisualization3}, a graphical specification environment for LTL specifications using touchscreen technologies is proposed. In \cite{LTLVisualization4}, the BPMN notation is extended to include descriptions of LTL rule to be verified.

As mentioned previously, we anticipate this work being useful in conjunction with hierarchical verification in situations where sub-processes or call activities are used. Hierarchical verification has been studied and used in many papers. In \cite{hierarchicalVerification1}, a compositional verification approach called Assume-guarantee reasoning is introduced and demonstrated on a simplified train control system. In \cite{hierarchicalVerification2}, the authors use edge-valued binary decision diagrams to enable multilevel and hierarchical verification of simple designs. In \cite{hierarchicalVerification3}, the authors use BPMN partitioning and colored Petri nets to enable a hierarchical verification technique for the state space analysis of BPMN workflows. In \cite{hierarchicalVerification4}, \cite{hierarchicalVerification6} and \cite{hierarchicalVerification7}, the authors develop and implement theories of hierarchical verification of speed-independent, asynchronous, and Galois field circuits, where specifications at one level of abstractions can be used as descriptions at higher levels of abstraction. In \cite{hierarchicalVerification5}, the authors introduce hierarchical concepts into artifact verification.


\section{Conclusions and Future Work}
\begin{comment}

The CWP was coupled with workflow models to create a verification problem suitable for the SPIN model checker. This was accomplished by translating the CWP into an set of LTL properties that together express the same meaning as the CWP. The workflow models can be directly turned into equivalent models in Promela, the input language to SPIN, which is able to prove whether the workflows correctly implement the CWP. Such mechanized and automated reasoning is critical to assurance that the design of a complex distributed system is capable to establish and maintain \emph{actionable risk awareness}. 

The CWP defines thresholds of patient risk during home care and their appropriate actions, thereby making a clear connection from the verified design to its larger, societal purpose of safe care. The interaction between users and AI is only modestly complex, however it strikes the balance needed to illustrate the method with an example workflow that can be followed by readers without much BPMN familiarity. 
The complexity should increase when deployment collects confirmation/dismissal of alerts, and eventual patient outcomes, for AI machine learning to increase alert precision.

Our approach's generality depends on discovery and modeling CWPs. Prior to model-checking CWPs their principles  were applied to human-computer integration for highly usable designs of integrated aircraft mission and maintenance scheduling \cite{workcentered}, joint U.S.-Russia maneuver planning for the \emph{International Space Station}  \cite{10.1145/1978942.1979311}, and health care coordination \cite{BERRY201615}. These CWPs define fundamental requirements for systems that must respond to events outside their direct control. The principles of CWP were recently adopted in the SysML v.2 standard.

Model-checking scalability depends on workflows' number of asynchronous actors, synchronization points, etc. BPMN supports hierarchies of sub-processes for larger workflows, where CWPs may be defined for each; regardless, the case study is a key step towards automation. 

Our current research aims to automate CWP translation to LTL, and BPMN to Promela, so counter-examples from SPIN point directly to CWPs. This undertaking is a non-trivial engineering task as model checking requires domain expertise and the counter-examples produced by SPIN are infinite by virtue of LTL semantics. Ongoing efforts are refining the SPIN translations to include additional global states that correspond directly to CWP states and edges to simplify counter-example mapping. 

Other research under consideration could explore reuse of verified designs in watcher systems that monitor their deployed implementations. CWP models could also be used to derive efficiency measures, e.g., reducing the amount of activity that does not advance the CWP states could be a new approach to efficiency.  

The focus of conventional design methods is on software, which leaves important aspects of system success largely to chance. 
This paper's novel contribution is verifiable integration of human-computer teaming on cognitive tasks. 
The need is industry-wide. 
Expected benefits of verifiable integration include greater safety and reliability of systems for many other critical domains. 

\end{comment}

Workflows have become a regular tool in modern business practices. They have also been increasingly used in the field of LC/NC development. But the efforts to verify workflow execution have been lacking. 

In this paper, we showed that workflows can be translated into verifiable Promela code. That code can then be verified against LTL properties represented by CWPs. This process can and has been automated, and we have provided an implementation for performing automated translation. Our solution uses an intermediate python graph structure and a visitor pattern to generate the Promela model and LTL properties. Additionally, we improve upon the format of the verification results by providing step-by-step annotated images when a verification error occurs. This improvement speeds up and eases the incremental process of modeling correct workflows.

The automated process of verifying BPMN workflows has been tested on three examples as case-studies and two more as scalability exercises. The results show that this method of automated model checking is viable for sizes of workflows frequently seen in LC/NC applications.

There are improvements to be made to the methods described in this paper. First, there should be efforts to conduct a usability study of the automation tool. We are interested not only in the ease of verification, but also in the generated Promela model's readability. Additionally, a usability study should seek to understand the usability of the generated counterexample in finding and fixing BPMN errors.

There should be an effort to further automate the identification of the behavior models. Currently, the behavior models must be manually written by an expert in the workflow's domain. It is our desire that the majority of the behavior models be inferred from different aspects of the BPMN itself. An iterative verification approach could be used: Auto-generated behavior models are tested through verification. After each iteration, verification results are analyzed and behavior models are modified. Additional tools such as AI might also be utilized.

There might also be efforts to explore additional verification properties of interest. For example, there may be a property that enforces eventual existence of valid end states in the BPMN. It may be the case that the BPMN accomplishes the work in the CWP before it reaches an end state itself. This might be undesirable behavior depending on the context.

In order to be usable on many commercial workflows, the tool will need to support many new BPMN elements, especially sub-tasks and call activities. For each of these, there should be support for hierarchical verification. Essentially, the sub-task should be verified independently to behave in the desired way, with its own associated CWP. Then, once we are confident it behaved that way, we can assume that behavior in the behavior models of the parent BPMN workflow. This process has not been implemented into the verification tool at this time.

We would like to compile a larger set of practical test inputs in order to get a more complete view of the tool's limitations and commercial viability. Commercial best modeling practices indicate that workflow scale should not cause problems for the automation tool, but having a large set of working examples would still be desirable.

There may be more effective alternative ways of representing the verification properties. Currently, we are representing them as LTL properties which are supported natively by the model checker we use, SPIN. It may be more efficient, however, to generate never claims directly, since SPIN internally converts each LTL property into never claims anyway.


\section*{Acknowledgment}
\addcontentsline{toc}{section}{Acknowledgment}

\begin{appendices}

\section{Evaluation Examples}
\label{sec:evaluationExamples}
The error discussed in \secref{sec:simpleExample} can be remedied in many ways. One way is by changing the workflow to initially negotiate \emph{price} and \emph{terms} separately. This change is shown in \figref{fig:face2face_workflow_fixed}. In the new workflow, only one of either \emph{price} or \emph{terms} changes at a time, making it impossible to transition directly from \emph{Init Purchase Pending} to \emph{Purchase Agreed}.

This problem could also be addressed by altering the CWP in a way that makes negotiations optional. Adding a transition from \emph{Init Purchase Pending} to \emph{Purchase Agreed} is a straightforward way of doing that.

\begin{figure*}[t]
  \begin{center}
    \begin{tabular}{c}
        \includegraphics[width=\textwidth]{../figs/BPMN/face2face_May_5_2023_workflow_fixed.png}
    \end{tabular}
  \end{center}
\caption{Fixed BPMN Workflow for \facetoface}
\label{fig:face2face_workflow_fixed}
\end{figure*}

\figref{fig:buynsell_bpmn} shows the BPMN of a remote purchase system named \buynsell. This BPMN model is similar to \facetoface~except that it splits the buyer and seller into different parties, as well as introduces some reasoning about fulfillment of the order. \buynsell shares a CWP with the previous example, shown in \figref{fig:PurchaseCWP}. This example was chosen partially to show the versatility of CWP. Because it shouldn't reason at all about the context or the actors in the work, it can be used to verify multiple BPMN diagrams involving an exchange of goods, even though they may have drastically different performance characteristics. Additionally, this example introduces asynchrony by dividing the work between two actors in separate swimlanes. This asynchrony is where manual reasoning becomes more difficult, and where the value of model checking becomes more apparent.

\begin{figure*}[t]
  \begin{center}
    \begin{tabular}{c}
        \includesvg[inkscapeformat=png, width=\textwidth]{../figs/BPMN/buynsell_Mar_27_2023_workflow.svg}
    \end{tabular}
  \end{center}
\caption{BPMN workflow for remote purchase example, \buynsell}
\label{fig:buynsell_bpmn}
\end{figure*}

\figref{fig:phware_bpmn} shows the BPMN of the original example from \cite{mercer22}. In this example, a patient diagnosed with COVID-19 is being monitored remotely. \figref{fig:phware_cwp} shows the corresponding CWP. When we verify the BPMN against this CWP, we are ensuring that the patient is never incidentally mistreated in the BPMN according to the readings from the remote monitoring device. The initial example shows a single actor with branching paths. The second example shows multiple actors with mostly linear paths. This is the most complicated example so far, containing both multiple actors and branching paths.

\begin{figure*}[t]
  \begin{center}
    \begin{tabular}{c}
        \includesvg[inkscapeformat=png, width=\textwidth]{../figs/BPMN/phware_May_5_workflow.svg}
    \end{tabular}
  \end{center}
\caption{BPMN workflow for \phware~example}
\label{fig:phware_bpmn}
\end{figure*}

\begin{figure*}[t]
  \begin{center}
    \begin{tabular}{c}
        \includegraphics[width=\textwidth]{../figs/CWP/phware_CWP.png}
    \end{tabular}
  \end{center}
\caption{CWP state diagram for \phware~example}
\label{fig:phware_cwp}
\end{figure*}

\figref{fig:ParallelScaling} shows a general version of the sequential scaling example BPMN and CWP. In this example, we show the scalability of the automation and verification process. This example contains a single actor with multiple decisions to make sequentially.
This number of decisions can be increased to push the limits of model checking in this context. The CWP here has just two states, only requiring that the BPMN execute to completion.


\figref{fig:SequentialScaling} shows the BPMN and CWP of the parallel scaling example. This is also an example to show scalability. In this example, there are multiple actors, each with a single decision to make. It is important to note that all of the actors interact with a single global state variable here. The CWP used in this example is identical to the sequential scaling example.

\begin{figure*}[t]
  \begin{center}
    \begin{tabular}{c}
        \includegraphics[width=\textwidth]{../figs/BPMN/ParallelScalingWorkflow.png} \\ 
        \includegraphics[width=\textwidth]{../figs/CWP/general_scaling_cwp.png}
    \end{tabular}
  \end{center}
\caption{BPMN workflow for parallel scaling with n actors}
\label{fig:ParallelScaling}
\end{figure*}


\begin{figure*}[t]
  \begin{center}
    \begin{tabular}{c}
        \includegraphics[width=\textwidth]{../figs/BPMN/SequentialScalingWorkflow.png} \\
        \includegraphics[width=\textwidth]{../figs/CWP/general_scaling_cwp.png}
    \end{tabular}
  \end{center}
\caption{BPMN workflow for sequential scaling}
\label{fig:SequentialScaling}
\end{figure*}


\section{Description of Modules}
\label{sec:descriptionOfModules}
The module BPMN.BPMN defines the intermediate python graph representation for a BPMN workflow. Each node in the graph also includes methods necessary for us to utilize the visitor pattern on it.

The module BPMN\_Visitor.BPMN\_Visitor defines the abstract BPMN visitor class. We inherit from this class to make visitors that generate Promela code.

The module XMLIngest.BPMNXMLIngestor is the utility for ingesting BPMN XML files and generating a python BPMN graph.

The module CWP.CWP defines the intermediate python graph representation for a CWP state diagram.

The module XMLIngest.CWPXMLIngestor is the utility for ingesting CWP XML files and generating a python CWP graph.

The module StateIngest.StateIngestor is the utility for ingesting a state TXT file written in the format described in \secref{sec:objectStateDefinition}. This module also contains the data class for a state variable. The result of ingesting the state file is a list of state variables.

The module StateIngest.ActivityModifiesIngest is currently unused.

The module ExpressionParse.ExpressionParser defines a class for parsing expressions. It is used in both the BPMN ingestion and in the CWP ingestion for both semantic and syntactic checks.

The module CSVIngest.CSVIngestor is a deprecated ingestion method for CWP diagrams.

The module PromelaGeneration.LTL\_gen includes a class for iterating through the elements of a CWP diagram and generating the corresponding LTL properties.

The module PromelaGeneration.Promela\_gen\_visitor is a concrete implementation of the BPMN\_visitor class defined earlier. This visitor traverses a BPMN graph and generates the corresponding Promela model for it. 

The module PromelaGeneration.Stub\_gen\_visitor is another concrete implementation of the BPMN\_visitor class. This visitor is used to traverse a BPMN and generate the skeleton of a behavior model inline for each element.

The module CounterExampleParser.CounterExampleParser contains all of the utility for parsing through the output of a simulation run following a trail file. This module also defines a CounterExampleStep class, which includes important information about one step in the model's execution. The result of parsing is a list of steps.

The module CounterExampleVisualize.BPMNETModifier contains the utility for modifying the XML files of a BPMN workflow based on a given CounterExampleStep.

The module CounterExampleVisualize.CWPETModifier contains the utility for modifying the XML file of a CWP state diagram based on a given CounterExampleStep.

The module CounterExampleVisualize.CounterExampleXMLGenerator contains the class that repeatedly uses the XML modifiers to generate a series of images, assisting in error visualization.

The module BPMN-Generate.LongScaling contains a class for generating a BPMN file for an example used in sequential scalability testing.

The module BPMN-Generate.WideScaling contains a class for generating a BPMN file for an example used in parallel scalability testing.

The module CWP-Generate.LongScaling  contains a class for generating a CWP file for an example used in sequential scalability testing.

The module CWP-Generate.WideScaling contains a class for generating a CWP file for an example used in parallel scalability testing.

Main.py is the CLI entrypoint into the project. One of the most important command line arguments for the project is -e, which is followed by the name of an example CWP-BPMN pair. This will generate the LTL and Promela, perform verification, and generate error visualization for the given example. The automation tool will check for input files in the directory code/assets/examples/\{exampleName\}, where exampleName is the name of the example given at the command line. In this directory there is expected to be several files, each of which has been discussed previously:
\begin{itemize}
\item A BPMN XML file (named workflow.bpmn)
\item A CWP XML file (named cwp.xml)
\item An object state file (named state.txt)
\item A behavior model Promela file (named behaviorModels.pml)
\end{itemize}

As the example is verifying, results are printed to the console and also written to a file, results.txt. This file is stored in code/output/examples/\{exampleName\}. This is also where the raw Promela model is stored. As mentioned earlier, a goal of this project was to make the generated model readable by one familiar with Promela. In some cases, careful inspection of the model may prove useful. Additionally, this is where any error visualization files are stored, under counterExamples/\{propertyName\}. A subdirectory will be created for each non-existential property that fails. Here, you will find XML and image files for both the BPMN and CWP. There will also be a PDF file for each, compiling all the image files together. Finally, when verification is finished, the tool writes a file named timing.txt that gives the time elapsed for promela model generation, model verification, and counterexample image generation.


\end{appendices}

\bibliographystyle{IEEEtran}
{\small 
\bibliography{bibliography}}

\end{document}
