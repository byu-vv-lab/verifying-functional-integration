
The BPMN diagrams are ingested into the translation tool as .bpmn files, which are in the format of xml. This format was standardized by Camunda, a Business Process Management company and investor in the BPMN standard. Diagrams can be exported into .bpmn using either the open-source online editor at bpmn.io, or Camunda's free desktop modeller. Both of these tools use the open-source project bpmn-js as a backend.

In order to facilitate easy addition of attributes and further distinguishing of BPMN elements, I am using an intermediate python representation before generating any Promela code. This Python representation is a tree structure, where each BPMN element is an object in the tree. Flows in the diagram connect elements as parents and children. This structure allows for easy traversal of the diagram.

I also added multiple attributes to each element in the diagram. One of these holds a boolean value for whether or not it has been seen already during a traversal. This helps avoid infinite looping during traversal. Another attribute is a Promela-friendly name derived from both the element type and its visible label. This name will be used to identify the element in the resulting Promela model. This role could be played by the "id" attribute already in existence in the .bpmn xml file, but this would greatly reduce readability of the Promela model, which was a point of interest throughout the project.

There are some situations in BPMN where the function of an element is implied by the structure in which it is used. For example, gateways can either be forking gateways (with multiple outgoing flows) or joining gateways (with multiple incoming flows). This distinction is not addressed directly in the BPMN syntax. I handle this by splitting some traditional BPMN elements into multiple elements in the python intermediate representation. There is, for example, both a "parallelGatewayJoinNode" class and a "parallelGatewayForkNode" class. This is important because the behavior of join and fork parallel gateways is significantly different.

Some of the labels in a BPMN diagram carry significant information necessary to the flow control. For example, some outgoing flows from an XOR gateway carry the logical condition under which that flow should be taken. When automating a system to generate Promela from a BPMN diagram, it is important to reason about which labels carry functional information and which are just informative. Additionally, there must be some demand on the content of functional labels. For conditions, the automation tool demands syntactically correct Promela boolean logic. If the structure of a BPMN determines that a flow's label is functional and that label is not in valid boolean logic, the automation tool will throw an exception.

