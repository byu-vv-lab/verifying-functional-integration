Another difficult part of automating is the generation of an environment. The initial definition of the environment must be provided to the automation tool as a set of variables and constants. Every variable that shows up in a conditional edge in the BPMN must be defined at this stage, otherwise verification will cease.

These environment variables are manipulated during the activities in the BPMN. The exact nature of these manipulations plays a vital role in the verification process. We call these manipulations the behavior models.

The behavior models are defined by a process beginning with specifying for each variable a set of possible values and a set of activities in which it can be modified. Then, it is the author's responsibility to further restrict the behavior models according to the complexity of the decisions in the workflow.

It should be noted that initial variable values are essential for the verification process. This is because the verifier will first check the LTL property, then step forward in the Promela code. This means that variables cannot be initialized in the first activity of the BPMN unless uninitialized variables are accounted for in the CWP. We recommend avoiding this and declaring initial values for at least some of the environment variables.

\egm{Which of the two above is implemented right now? Let's write about that one and indicate the other as future work?}