\begin{comment}
Automatic translation from CWP and BPMN to LTL properties and Promela code, respectively, is not trivial. First, the edges of the CWP must follow a consistent expression language in order to be interpreted automatically. Next, the object state must be clearly defined. Also, several implied BPMN semantics lead to ambiguous workflow structures, requiring additional automated reasoning. Next, the complete conditions of decision gateways in the workflow must be expressed somewhere. Finally, the environment for running the workflow must be defined, including variables and behavior models for BPMN activities.
\end{comment}

\figref{fig:NewSolutionRoadmap} shows a diagram of our improved solution for verifying BPMNs against CWPs. It is an improvement upon the existing solution described in \secref{sec:background}. Our additions to the solution are colored teal and dashed. One of the primary advantages of the new solution is that almost the entire translation and verification processes are automated.

One noteworthy goal of our automated generation was to create idiomatic code that is relatively understandable by someone familiar with the Promela language. We made this a goal because some errors can be diagnosed and solved most quickly by direct inspection of the Promela model. Therefore, there is overhead, some of which is found in code snippets throughout the paper, to make the generated Promela more readable. For example, we use parts of task labels in their unique identifiers (as opposed to standard BPMN encoding, which assigns random alphanumeric strings.) Additionally, because we don't have access to a code formatter for Promela, we keep track of indentation and whitespace throughout the code generation process. Even though these factors are not critically important to our solution's technical success, we maintain that it is a significant goal. One important piece of future work is a user study evaluating the readability of the code generated as part of this project.

The first addition is an input file that provides a list of variables, their initial values, and their domains. This helps define the object state, which is a unique state of execution for the BPMN. The next addition is another input file that provides details on the manner in which the workflow modifies state variables at each of its activities. Details on each of the additional inputs are given in \secref{sec:objectStateDefinition} and \secref{sec:behaviorModels}

The next two additions are the LTL generation and the Promela generation. Previously, these two steps were performed manually. In this tool, both are automated. These automation processes are discussed in both \secref{sec:ltlGeneration} and \secref{sec:promelaGeneration}. More details on the architecture of the code generation tools are given later in \secref{sec:toolArchitecture}.
