\begin{comment}

This paper introduces a new method of \emph{model based systems engineering} (MBSE) for functional integration and verification of joint human-computer teaming on cognitive tasks. 
The original contribution to integration is a specification of the abstract, cognitive work problem (CWP) the design must solve \cite{workcentered,BERRY201615,chi2010}, represented as a declarative, computation independent model \cite{Garrido}.
The technical neutrality of this cognitive model makes it a shareable object of abstract work which different types of actors can transform, despite vastly different performance properties.
Thus, a CWP provides the means to integrate the work of a distributed cognitive system that transforms a CWP from its initial state to the goal state required from the system.

The original contribution to design verification reuses the CWP as a new type of required property for model checking to prove the correctness of integrated designs.
The capability to integrate human cognition with machine reasoning and then verify design correctness can increase safety and reliability for a wide range of critical systems.

Our proof-of-technology is illustrated by a case-study of telehealth augmented by \emph{remote patient monitoring} (RPM).
Although telehealth has rapidly gained acceptance for improving accessibility \cite{Medicare} and reducing transmission rates \cite{10.1093/jamia/ocaa067} the remote and asynchronous context of telehealth also introduces risk that patients may deteriorate before physicians can be aware and intervene \cite{10.1097/ALN.0000000000003578}.
The experience of this integrated design process makes clear the difficulty of reasoning about even simple systems without a verification tool.

Bionous \phware\textsuperscript{\textregistered} is an application under development to enhance the telehealth safety of COVID-19 patients with RPM of their vital signs \cite{Phware}.
\phware\ measures and streams seven vital signs: temperature, oxygen saturation, respiration rate, pulse and its variability, and dia/systolic blood pressures.
A single \phware\ ring or finger-clip can be used at home to sense all vital signs in a 60-90 second session, and then send them to a smartphone via Bluetooth.
A smartphone app uploads the data to AI cloud services for storage and machine-learning to assign numerical values to vitals, analyze their trends, and send alerts.
A web-based dashboard displays the trends and alerts to provide clinicians with \emph{actionable risk awareness} of their outpatients.
The streaming of vital signs and alerts serve as an early warning system to maintain clinicians’ awareness and prioritize their attention, thereby enhancing home care safety.

The following section defines \emph{actionable risk awareness} as the CWP for \phware, developed with clinicians and other experts \footnote{Ann Marie Kimball, an internationally recognized MD epidemiologist in emerging infectious disease, provided medical advice}.
\secref{sec:ltl} details translation of the CWP into \emph{Linear Temporal Logic} (LTL) \cite{10.5555/975331} properties that exactly capture the meaning of the CWP and are suitable for use with a model checker.
%LTL is a first-order predicate calculus with the ability to reason about the temporal ordering of events \cite{10.5555/975331}.
%Any viable design for vital sign RPM of COVID-19 patients must verify against each LTL property created from the CWP in order to claim it takes appropriate action relative to patient risk.
\secref{sec:workflow} presents the \phware\ workflow model in the \emph{Business Process Modeling Notation} (BPMN) \cite{BPMN} that integrates the activities of clinicians, AI cloud services, and patient-caregivers.
\secref{sec:bpmn} details the translation of the workflow into an equivalent \emph{Process Meta-language} (Promela) model for the SPIN model-checker \cite{spin} with \secref{sec:env} discussing how the inputs to the workflow are defined.
The results of the SPIN verification are given in \secref{sec:results}.

The original finite state model, the BPMN model, and full Promela model, with the LTL properties for the CWP, are in a public Github repository \cite{repo}.
The \texttt{README.md} file summarizes its content and how to use SPIN to verify the model.
Figure titles in the electronic version of this manuscript link to equivalent, but larger, more legible versions in GitHub.


% The intentions of the detailed case study are to show the viability of augmenting workflows with CWP for functional integration and model checking, and to illustrate the value for automated reasoning in any design environment.
% SPIN worked effectively and efficiently in this project in terms of performance and iterative design analysis.
% The workflows and CWP revisions were easy to verify when posted, and when SPIN found violations, the accompanying counter-examples provided critical insights to the unexpected, and often overlooked, outcomes common in asynchronous interactions.
% These insights were absolutely necessary for those working on the workflows and those working on the CWP to arrive at the presented final forms.
% The experience of this integrated design process with the model checker makes clear that it is difficult to reliably reason about even simple concurrent systems without a verification tool.
\end{comment}

This paper introduces a novel approach to automating model checking of systems involving integration of human cognition and machine reasoning. We build upon prior work \cite{mercer22} introducing a new method for \emph{model based systems engineering} (MBSE) . A core component to functional integration is an abstract specification of work, a Cognitive Work Problem (CWP). A CWP is a computation-agnostic model, allowing it to be operated on by any actor, regardless of their performance properties. The work in a distributed cognitive system is integrated if it transforms a CWP from its initial state to a goal state.

Humans and machines work together in practically every field. In may of these fields, seamless integration is not only desired, but critical. There are safety concerns, such as in aeronautics\cite{aeronautics1, aeronautics2} or healthcare\cite{healthcare1, healthcare2}. There are financial concerns, such as in banking\cite{banking1,banking2}. There are even concerns about integrity, such as in voting\cite{voting1, voting2}.
\egm{Add citations to the above text justify the claims where possible. Keith may have such citations or they may exist in the paper that first introduces the translation.}

This work progresses the path toward seamless integration between humans and machines in business processes. It does that by partially automating a process which allows Business Process Model and Notation (BPMN)\cite{BPMN} models to be verified against the CWP, a separate model which is taken as ground truth. The CWP has two parts, an object state and a state machine; the former connects it to the BPMN workflow model. The goal of the verification process is to ensure that the BPMN workflow can manipulate the object state toward any of the final states specified by the CWP state machine. Verification must also ensure that the BPMN workflow uses only allowed transitions to reach a final state. So that the process may benefit human-computer interaction, it is critical that the CWP should not reason about the environment in which the work gets done, the context in which the work gets done, or the actors who perform the work.
\egm{BPMN needs to be defined. Also, make clear that the separate model is the CWP. It may also be wise to refer to what makes up a CWP, object state and a state machine, and how the CWP and workflow are related to one another by virtue of that object state. The goal or the workflow is to evolve the object state to any of the allowed final states in a ways that always follows transitions allowed by the CWP.}

Verification of functional integration has been attempted before using task decomposition \cite{decomposition}. There are also examples verifying properties of BPMN workflows.\cite{bpmnVerification1, bpmnVerification2, ASMverification} These methods verify properties such as absence of deadlock, fair resource usage, and absence of invalid end states.
As mentioned earlier, this work is primarily built upon the work in \cite{mercer22}, which creates a foundation for functional integrity of human cognition and machine reasoning using model checking. Our work borrows much of the structural designs of both Linear Temporal Logic (LTL)\cite{ltl} property generation and also Process Meta-Language (Promela)\cite{promelaManual} model generation, while adding automation and counterexample visualization.
Prior work in verifying BPMN models can verify certain global properties, such as absence of deadlock or the existence of an end state, but this work adds a ground truth model which introduces verification of the semantics of the BPMN model as well as the structure. 
\egm{This paragraph needs to expand, and actually cite, the related work. There is work on task decomposition as a way to do functional integration. There is my work with Keith that is published that must be discussed here since this paper needs to be precise about what it does new, or what problems my previous paper left unsolved (automation and counter-example explanation). There is also work on verifying BMPN message diagrams with SPIN and some work on verifying mode confusion on graphical interfaces. There should be maybe two paragraphes here: the first is related stuff with other ways to do functional integration and the second is my work that is better than that in the first paragraph but still leaves unsolved automation and counter-example explanation.}

\egm{Missing is a clear statement about our new solution to the problem and what it is that that solution accomplishes that my prior work does not do. It should also state how we intend to show that our new solution works and solves the problems that it claims to solve. After that paragraph, add a paragraph that list the contributions of the research discussed in this paper.}

\secref{sec:simpleExample} provides a simple example of a BPMN-CWP pair and shows how model checking can allow us to prove correct implementation.
\secref{sec:background} gives a background of \cite{mercer22} and how it relates to the work in this paper.
\secref{sec:automaticGeneration} looks at the difficulties of automating the Promela generation.
\secref{sec:counterexample} introduces the counterexample visualization tool, which improves significantly on the prior methods of debugging BPMN workflow errors.
\secref{sec:eval} provides evaluation characteristics, as well as the set of examples this tool is evaluated on, explaining the unique aspects of each example, as well as providing performance statistics.