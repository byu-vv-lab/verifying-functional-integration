We explored how the CWP states are defined as boolean expressions using the edges as atomic propositions in \secref{sec:backgroundCWP}. Therefore, in order to automate the process, the edges must be machine-readable boolean logic expressions. For simplicity we chose to write the edges in valid Promela syntax. Promela boolean operators follow a style popularized by C, where "<", "<=", ">", ">=", "==" and "!=" represent "less than", "less than or equal", "greater than", "greater than or equal", "equal" and "not equal" respectively.0

There are only two constants allowable in the boolean expression, "True" and "False". Any other constant is meaningless. For example, the expression "PriorityLevel <= 5" has no significant meaning, even if an existing standard of priority levels is well established in the corresponding BPMN. The reasoning for this is explained further in \secref{sec:envConstruction}.

The expression is written visually on the diagram as an edge label. This has the advantage of being completely transparent to the viewer, along with the downside that it may be unreadable to readers lacking a background in boolean logic. This can be improved in the future by using a more user-friendly expression language such as the "Friendly Enough Expression Language" (FEEL) *CITATION NEEDED*, and by using an additional non-visual XML attribute that carries the expression separate from the edge label itself.

There was another problem with the CWP-to-LTL translation that should be discussed here; because each property searches only fair paths (paths that eventually transition the object state of the CWP to a goal state), if there is no fair path, every property searches no paths and becomes vacuously true. Even if there are other errors in the workflow (certain states do not exist, improper transitions are used, etc) they will get overshadowed by the absence of a fair path. This is especially problematic because, in many situations, the additional errors contribute to the absence of a fair path. The solution I used is simple: Add a runtime flag that removes the fair precondition from each property. This normally isn't desireable, as paths that do not end in a goal state are of no interest during formal verification. Removing the precondition is only a method of adding debugging capabilities when the model is already failing verification (by lacking a fair path). The option should not be used if a fair path exists in the model.

