Because of the relative simplicity and familiarity of a purchase or exchange, we chose the following CWP and BPMN pair to exemplify the process of integration using model checking. The CWP, shown in figure 1, has five states, of which 2 are goal states. The workflow is responsible for transforming the object state to the goal state using only valid transitions as defined by the CWP.

The CWP defines five distinct states. The first is in the top left corner, where the purchase has been initialized. Next, there is a state near the center defining negotiations. There are then states for either the purchase failing or being agreed upon. Finally, there is a state at the bottom right for the ownerships switching.

The BPMN workflow begins on the left with the start event "Start7." The flow then takes the token through task one, where the buyer and seller meet. Then, the parallel gateways indicate that task 2 and task 3 are performed in parallel. This is where negotiations are done for the price and terms of the purchase. Next, an exclusive OR gateway determines whether the negotiations were successful. If they were, and both parties agreed on price and terms, the flow can continue to task 7, where they shake hands, and then further on to tasks 8a and 8b where the payment and the backpack are exchanged. If negotiations were unsuccessful initially, the XOR gateway will send the flow back to task 6, where both parties regroup. After this, an additional XOR gateway determines whether negotiations should be repeated for price (to task 4), terms (to task 5), or both (to tasks 2 and 3).

\bryce{I am really struggling to see a coherent way of introducing the error. Should I first introduce the flawed example? It just seems silly for me to introduce an example (with associated walkthrough of the CWP and BPMN) by showing it's flawed version first. But I may be wrong.}