\begin{comment}
Functional integration of human cognition and machine reasoning is an industry-wide problem where failure risks health or safety.
Differences in human versus machine functioning obscure conventional integration. 
We introduce cognitive work problems (CWP) for rigorous, verifiable functional integration. 
CWP specify the cognitive problem that integrated designs must solve. 
They are technology-neutral, abstract work objects, allowing people and computing to share and transform them in coordination.
The end-to-end method is illustrated on a system that employs AI for remote patient monitoring (RPM) during COVID-19 home care. 
The CWP specified \emph{actionable risk awareness} as the medical problem RPM must solve.
Graphical modeling standards enabled user participation: CWP as finite state machines and system behavior in BPMN. 
For model checking, the CWP’s logical content was translated to linear temporal logic (LTL) and the BPMN into Promela as inputs to the SPIN model checker. 
SPIN verified the Promela implements the LTL correctly.
We conclude this CWP-derived RPM design solves the medical problem and enhances patient safety. 
The method appears general to many critical systems.

\end{comment}

Many critical systems industry-wide demand fail-safe integration of human cognition and machine reasoning.
This integration is difficult because of inherent dissimilarity between human and machine functioning.
Existing methods for validating systems with human and machine functioning require expertise in formal verification and are labor intensive.
Additionally, error output from existing methods is exclusively text-based and demands extensive time and familiarity to comprehend.
We introduce a tool for automating verification of systems with human and machine functioning.
Participation in the automation process requires very little knowledge of the underlying model checking tools.
We demonstrate the effectiveness of the tool using examples of models including subtle-but-critical errors.
We present an extensible model to demonstrate the scalability of the automated verification tool.
We also introduce an understandable visual representation of model failures to enable quick iteration on subsequent model versions.
The visual representation is critical to understanding why a model doesn't comply with given temporal properties.
We conclude that model checking is a viable method for verifying systems that require human and machine integration.
